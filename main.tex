%%%%%%%%%%%%%%%%%%%%%%%%%%%%%%%%%%%%%%%%%%%%%%%%%%%%%%%%%%%%%%%%%%%%%%%%%%%%%%%%
\documentclass[12pt,UTF8,AutoFakeBold=3,a4paper]{article} 
%%%%%%%%%%%%%%%%%%%%%%%%%%%%%%%%%%%%%%%%%%%%%%%%%%%%%%%%%%%%%%%%%%%%%%%%%%%%%%%%

\usepackage{ctex}

\usepackage{geometry} %改改尺寸
\geometry{
  left=1.25in,
  right=1.25in,
  top=1in,
  bottom=1in
}

%Ms Word 的蓝色和latex xcolor包预定义的蓝色不一样。通过屏幕取色得到。
\usepackage{xcolor}
\definecolor{NsfcBlue}{RGB}{0,112,192}

\usepackage[unicode, hidelinks]{hyperref} %提供跳转链接

\usepackage[english]{babel} %支持混合语言

\usepackage{graphicx} 
\graphicspath{{figs/}}     

\usepackage[labelfont=bf]{caption}
\addto\captionsenglish{
    \renewcommand{\contentsname}{目录}
    \renewcommand{\listfigurename}{插图目录}
    \renewcommand{\listtablename}{表格}
    %\renewcommand{\refname}{\sihao 参考文献}
    % 这几个字默认字号稍大,改成四号字,楷书,居左(默认居中) 根据喜好自行修改,官方模板未作要求
    \renewcommand{\refname}{\sihao \kaishu \leftline{参考文献}}
    \renewcommand{\abstractname}{摘要}
    \renewcommand{\indexname}{索引}
    \renewcommand{\tablename}{表}
    \renewcommand{\figurename}{图}
} %把Figure改成‘图’,reference改成‘参考文献’。如此处理是为了避免和babel包冲突。
%定义字号

\usepackage{amsmath} %更多数学符号
\usepackage{wasysym}
\usepackage{ragged2e}
\usepackage{enumitem}

\usepackage{physics}
\usepackage{siunitx}


\usepackage[
    defernumbers=true,
    backend=biber,
    % sorting=ymdnt,                     % Year in descending order
    sorting=ynt,                      % Year in ascending order
    maxbibnames=3,                   % No. of listed names
    style=ieee,
    citestyle=numeric-comp,
    isbn=false,                       % controls whether the fields isbn/issn/isrn are printed
    % block=par,
    doi=false,                        % do not show doi
    giveninits=false,
]{biblatex}
\addbibresource{FundingRef.bib}

\DeclareSortingScheme{ymdnt}{
  \sort{
    \field{presort}
  }
  \sort[final]{
    \field{sortkey}
  }
  \sort[direction=descending]{
    \field[strside=left,strwidth=4]{sortyear}
    \field[strside=left,strwidth=4]{year}
    \literal{9999}
  }
  \sort[direction=descending]{
    \field{month}
    \literal{9999}
  }
  \sort{
    \field{sortname}
    \field{author}
    \field{editor}
    \field{translator}
    \field{sorttitle}
    \field{title}
  }
  \sort{
    \field{sorttitle}
    \field{title}
  }
}


%%%%%%%%%%%%%%%%%%%%%%%%%%%%%%%%%%%%%%%%%%%%%%%%%%%%%%%%%%%%%%%%%%%%%%%%%%%%%%%%
% Commands
%%%%%%%%%%%%%%%%%%%%%%%%%%%%%%%%%%%%%%%%%%%%%%%%%%%%%%%%%%%%%%%%%%%%%%%%%%%%%%%%
\newcommand{\chuhao}{\fontsize{42pt}{\baselineskip}\selectfont}
\newcommand{\xiaochuhao}{\fontsize{36pt}{\baselineskip}\selectfont}
\newcommand{\yihao}{\fontsize{26pt}{\baselineskip}\selectfont}
\newcommand{\erhao}{\fontsize{22pt}{\baselineskip}\selectfont}
\newcommand{\xiaoerhao}{\fontsize{18pt}{\baselineskip}\selectfont}
\newcommand{\sanhao}{\fontsize{16pt}{\baselineskip}\selectfont}
\newcommand{\sihao}{\fontsize{14pt}{\baselineskip}\selectfont}
\newcommand{\xiaosihao}{\fontsize{12pt}{\baselineskip}\selectfont}
\newcommand{\wuhao}{\fontsize{10.5pt}{\baselineskip}\selectfont}
\newcommand{\xiaowuhao}{\fontsize{9pt}{\baselineskip}\selectfont}
\newcommand{\liuhao}{\fontsize{7.875pt}{\baselineskip}\selectfont}
\newcommand{\qihao}{\fontsize{5.25pt}{\baselineskip}\selectfont}
%字号对照表
%二号 21pt
%四号 14
%小四 12
%五号 10.5
%设置行距 1.5倍
\renewcommand{\baselinestretch}{1.5}
\XeTeXlinebreaklocale "zh"           % 中文断行

\newcommand{\hamil}{{\cal H}}
\newcommand{\namdr}{NAMD\_\textbf{r}}
\newcommand{\namdk}{NAMD\_\textbf{k}}

%%%%%%%%%%%%%%%%%%%%%%%%%%%%%%%%%%%%%%%%%%%%%%%%%%%%%%%%%%%%%%%%%%%%%%%%%%%%%%%%
\pagestyle{empty}
%%%%%%%%%%%%%%%%%%%%%%%%%%%%%%%%%%%%%%%%%%%%%%%%%%%%%%%%%%%%%%%%%%%%%%%%%%%%%%%%
%%%% 正文开始 %%%%
%%%%%%%%%%%%%%%%%%%%%%%%%%%%%%%%%%%%%%%%%%%%%%%%%%%%%%%%%%%%%%%%%%%%%%%%%%%%%%%%
\begin{document}
%%%%%%%%%%%%%%%%%%%%%%%%%%%%%%%%%%%%%%%%%%%%%%%%%%%%%%%%%%%%%%%%%%%%%%%%%%%%%%%%

\begin{center}
 \kaishu{} \bfseries{} \sanhao{} 报告正文
\end{center}

{\kaishu{}\bfseries{}\sihao{}
  \textcolor{NsfcBlue}
  {(请勿删除或改动下述提纲标题及括号中的文字)}
}

%%%%%%%%%%%%%%%%%%%%%%%%%%%%%%%%%%%%%%%%%%%%%%%%%%%%%%%%%%%%%%%%%%%%%%%%%%%%%%%%
% Part 1
%%%%%%%%%%%%%%%%%%%%%%%%%%%%%%%%%%%%%%%%%%%%%%%%%%%%%%%%%%%%%%%%%%%%%%%%%%%%%%%%
{\kaishu{}\sihao{}
  \textcolor{NsfcBlue}
  {\bfseries{}(一)主要学术成绩、创新点及其科学意义}
  \color{NsfcBlue}(建议不超过4000字)
}

{\kaishu{}\sihao{}
  \textcolor{NsfcBlue}{
  按本年度《国家自然科学基金项目指南》中优秀青年科学基金项目的有关要求,着重阐述
  所取得的研究成果的创新性和科学价值等。
}}%

%%%%%%%%%%%%%%%%%%%%%%%%%%%%%%%%%%%%%%%%%%%%%%%%%%%%%%%%%%%%%%%%%%%%%%%%%%%%%%%%
\vspace{6pt}
%%%%%%%%%%%%%%%%%%%%%%%%%%%%%%%%%%%%%%%%%%%%%%%%%%%%%%%%%%%%%%%%%%%%%%%%%%%%%%%%


对于本项目四个问题的技术路径我们都做了公式的推导以及仔细的分析,并给出了清晰的执
行路线图,动量空间载流子动力学主要涉及到电声耦合矩阵元的计算以及大量电子态的演化,
激发态受力的计算涉及到 DFPT 与 BSE 计算,光场的耦合的关键物理量是跃迁偶极矩,摩擦
与碰撞的问题则只需修改细致平衡条件。我们已经对计算中关键步骤,例如电声耦
合、DFPT、BSE、跃迁偶极等模块进行了测试,有信心能够顺利完成项目目标。

%%%%%%%%%%%%%%%%%%%%%%%%%%%%%%%%%%%%%%%%%%%%%%%%%%%%%%%%%%%%%%%%%%%%%%%%%%%%%%%%
% Part 2
%%%%%%%%%%%%%%%%%%%%%%%%%%%%%%%%%%%%%%%%%%%%%%%%%%%%%%%%%%%%%%%%%%%%%%%%%%%%%%%%
{\kaishu{}\sihao{}
  \textcolor{NsfcBlue}
  {\bfseries{}(二)拟开展的研究工作}
  \color{NsfcBlue}(建议不超过4000字)
}

{\kaishu{}\sihao{}
  \color{NsfcBlue}
  着重阐述拟开展的研究工作的科学意义和创新性,技术路线、研究方案等的可行性。
}
%%%%%%%%%%%%%%%%%%%%%%%%%%%%%%%%%%%%%%%%%%%%%%%%%%%%%%%%%%%%%%%%%%%%%%%%%%%%%%%%

%%%%%%%%%%%%%%%%%%%%%%%%%%%%%%%%%%%%%%%%%%%%%%%%%%%%%%%%%%%%%%%%%%%%%%%%%%%%%%%%
% Part 3
%%%%%%%%%%%%%%%%%%%%%%%%%%%%%%%%%%%%%%%%%%%%%%%%%%%%%%%%%%%%%%%%%%%%%%%%%%%%%%%%
{\kaishu{}\sihao{}
  \textcolor{NsfcBlue}
  {\bfseries{}(三)其他需要说明的情况}
}


%%%%%%%%%%%%%%%%%%%%%%%%%%%%%%%%%%%%%%%%%%%%%%%%%%%%%%%%%%%%
{\kaishu{}\sihao{} \color{NsfcBlue}
1. 申请人同年申请不同类型的国家自然科学基金项目情况(列明同年申请的其他项目的项
目类型、项目名称信息,并说明与本项目之间的区别与联系)。
}
%%%%%%%%%%%%%%%%%%%%%%%%%%%%%%%%%%%%%%%%%%%%%%%%%%%%%%%%%%%%

%%%%%%%%%%%%%%%%%%%%%%%%%%%%%%%%%%%%%%%%%%%%%%%%%%%%%%%%%%%%
{\kaishu{}\sihao{} \color{NsfcBlue}
2. 具有高级专业技术职务(职称)的申请人是否存在同年申请或者参与申请国家自然科学
基金项目的单位不一致的情况;如存在上述情况,列明所涉及人员的姓名,申请或参与申请
的其他项目的项目类型、项目名称、单位名称、上述人员在该项目中是申请人还是参与者,
并说明单位不一致原因。
}
%%%%%%%%%%%%%%%%%%%%%%%%%%%%%%%%%%%%%%%%%%%%%%%%%%%%%%%%%%%%

%%%%%%%%%%%%%%%%%%%%%%%%%%%%%%%%%%%%%%%%%%%%%%%%%%%%%%%%%%%%
{\kaishu{}\sihao{} \color{NsfcBlue}
3. 具有高级专业技术职务(职称)的申请人是否存在与正在承担的国家自然科学基金项目
的单位不一致的情况;如存在上述情况,列明所涉及人员的姓名,正在承担项目的批准号、
项目类型、项目名称、单位名称、起止年月,并说明单位不一致原因。
}
%%%%%%%%%%%%%%%%%%%%%%%%%%%%%%%%%%%%%%%%%%%%%%%%%%%%%%%%%%%%

%%%%%%%%%%%%%%%%%%%%%%%%%%%%%%%%%%%%%%%%%%%%%%%%%%%%%%%%%%%%
{\kaishu{}\sihao{} \color{NsfcBlue}
4. 其他。
}
%%%%%%%%%%%%%%%%%%%%%%%%%%%%%%%%%%%%%%%%%%%%%%%%%%%%%%%%%%%%
%%%%%%%%%%%%%%%%%%%%%%%%%%%%%%%%%%%%%%%%%%%%%%%%%%%%%%%%%%%%%%%%%%%%%%%%%%%%%%%%

%%%%%%%%%%%%%%%%%%%%%%%%%%%%%%%%%%%%%%%%%%%%%%%%%%%%%%%%%%%%%%%%%%%%%%%%%%%%%%%%
\end{document}
%%%%%%%%%%%%%%%%%%%%%%%%%%%%%%%%%%%%%%%%%%%%%%%%%%%%%%%%%%%%%%%%%%%%%%%%%%%%%%%%


