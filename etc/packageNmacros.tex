\usepackage{ctex}

\usepackage{geometry} %改改尺寸
\geometry{
  left=1.25in,
  right=1.25in,
  top=1in,
  bottom=1in
}

\usepackage{tcolorbox}
\tcbuselibrary{
  most
}
\newtcolorbox{InnovationBox}[1][]{
    width=0.95\linewidth,
    title=科学贡献和创新,
    fonttitle=\xiaosihao\kaishu\bfseries,
    fontupper=\kaishu,
    colback=gray!10!white, colframe=EmphColor,
    colbacktitle=EmphColor,
    % colupper=EmphColor,
    boxrule=0.8pt,
    enhanced,
    attach boxed title to top left={
      xshift=12pt, yshift=-\tcboxedtitleheight/2,
      yshifttext=-10pt,
    },
    #1
}

%%%%%%%%%%%%%%%%%%%%%%%%%%%%%%%%%%%%%%%%%%%%%%%%%%%%%%%%%%%%
\usepackage{titlesec}

\titleformat*{\subsection}{\sihao\kaishu\bfseries{}\color{EmphColor}}

%Ms Word 的蓝色和latex xcolor包预定义的蓝色不一样。通过屏幕取色得到。
\usepackage[]{xcolor}
\definecolor{NsfcBlue}{RGB}{0, 112, 192}
\definecolor{EmphColor}{RGB}{218, 100, 20}

\usepackage[unicode, hidelinks]{hyperref} %提供跳转链接

\usepackage[english]{babel} %支持混合语言

\usepackage{graphicx} 
\graphicspath{{figs/}}     

\usepackage[labelfont=bf]{caption}
\addto\captionsenglish{
    \renewcommand{\contentsname}{目录}
    \renewcommand{\listfigurename}{插图目录}
    \renewcommand{\listtablename}{表格}
    %\renewcommand{\refname}{\sihao 参考文献}
    % 这几个字默认字号稍大,改成四号字,楷书,居左(默认居中) 根据喜好自行修改,官方模板未作要求
    \renewcommand{\refname}{\sihao \kaishu \leftline{参考文献}}
    \renewcommand{\abstractname}{摘要}
    \renewcommand{\indexname}{索引}
    \renewcommand{\tablename}{表}
    \renewcommand{\figurename}{\kaishu{}图}
} %把Figure改成‘图’,reference改成‘参考文献’。如此处理是为了避免和babel包冲突。
%定义字号
% \captionsetup[figure]{font=it}

\usepackage{amsmath} %更多数学符号
\usepackage{wasysym}
\usepackage{ragged2e}
\usepackage[inline]{enumitem}


\usepackage{physics}
\usepackage{siunitx}


\usepackage[
    defernumbers=true,
    backend=biber,
    % sorting=ymdnt,                     % Year in descending order
    sorting=ynt,                      % Year in ascending order
    maxbibnames=3,                   % No. of listed names
    style=ieee,
    citestyle=numeric-comp,
    isbn=false,                       % controls whether the fields isbn/issn/isrn are printed
    % block=par,
    doi=false,                        % do not show doi
    giveninits=false,
]{biblatex}
\addbibresource{FundingRef.bib}

\DeclareSortingScheme{ymdnt}{
  \sort{
    \field{presort}
  }
  \sort[final]{
    \field{sortkey}
  }
  \sort[direction=descending]{
    \field[strside=left,strwidth=4]{sortyear}
    \field[strside=left,strwidth=4]{year}
    \literal{9999}
  }
  \sort[direction=descending]{
    \field{month}
    \literal{9999}
  }
  \sort{
    \field{sortname}
    \field{author}
    \field{editor}
    \field{translator}
    \field{sorttitle}
    \field{title}
  }
  \sort{
    \field{sorttitle}
    \field{title}
  }
}


%%%%%%%%%%%%%%%%%%%%%%%%%%%%%%%%%%%%%%%%%%%%%%%%%%%%%%%%%%%%%%%%%%%%%%%%%%%%%%%%
% Commands
%%%%%%%%%%%%%%%%%%%%%%%%%%%%%%%%%%%%%%%%%%%%%%%%%%%%%%%%%%%%%%%%%%%%%%%%%%%%%%%%
\newcommand{\chuhao}{\fontsize{42pt}{\baselineskip}\selectfont}
\newcommand{\xiaochuhao}{\fontsize{36pt}{\baselineskip}\selectfont}
\newcommand{\yihao}{\fontsize{26pt}{\baselineskip}\selectfont}
\newcommand{\erhao}{\fontsize{22pt}{\baselineskip}\selectfont}
\newcommand{\xiaoerhao}{\fontsize{18pt}{\baselineskip}\selectfont}
\newcommand{\sanhao}{\fontsize{16pt}{\baselineskip}\selectfont}
\newcommand{\sihao}{\fontsize{14pt}{\baselineskip}\selectfont}
\newcommand{\xiaosihao}{\fontsize{12pt}{\baselineskip}\selectfont}
\newcommand{\wuhao}{\fontsize{10.5pt}{\baselineskip}\selectfont}
\newcommand{\xiaowuhao}{\fontsize{9pt}{\baselineskip}\selectfont}
\newcommand{\liuhao}{\fontsize{7.875pt}{\baselineskip}\selectfont}
\newcommand{\qihao}{\fontsize{5.25pt}{\baselineskip}\selectfont}
%字号对照表
%二号 21pt
%四号 14
%小四 12
%五号 10.5
%设置行距 1.5倍
\renewcommand{\baselinestretch}{1.5}
\XeTeXlinebreaklocale "zh"           % 中文断行
\setlength{\parskip}{6pt}

\DeclareEmphSequence{\bfseries\color{EmphColor}}
  
\newcommand{\namdk}{NAMD\_{\bf k}}
\newcommand{\hnamd}{Hefei-NAMD}
