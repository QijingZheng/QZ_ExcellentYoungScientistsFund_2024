%%%%%%%%%%%%%%%%%%%%%%%%%%%%%%%%%%%%%%%%%%%%%%%%%%%%%%%%%%%%%%%%%%%%%%%%%%%%%%%%
\setcounter{section}{0}

%%%%%%%%%%%%%%%%%%%%%%%%%%%%%%%%%%%%%%%%%%%%%%%%%%%%%%%%%%%%%%%%%%%%%%%%%%%%%%%%
\section{开展二维材料中的激子动力学理论研究}

\subsection{科学意义和创新性}

激子,作为一种光激发产生的准粒子,会在一定时间内衰减,高能量的激子会在一定的时间
内衰减到激子的基态上。在不同的光激发强度下,激子的弛豫通道会有所不同:在低强度激
发下,激子-声子散射是主要的弛豫通道;而在高强度激发下,激子-激子散射的作用则更加
明显。此外,其他因素也会影响到激子的弛豫过程,比如体系的缺陷、杂质、自旋轨道耦合
等。当激子弛豫到其基态时,电子和空穴会在一定时间内复合,其中只有一小部分可能通过
辐射发光的形式复合,大部分则通过非辐射跃迁的方式复合,将多余的能量以热的形式耗散
掉。\emph{激子动力学的问题无论是在基础科学层次还是在实际应用层面都具有重要的意义}。
比如激子的寿命是决定太阳能转化效率的关键因素,也决定了光电器件中响应速度的快慢;
具有第二类能带匹配的二维异质结中层内激子到层间激子转化的速度决定了电子与空穴分离
的快慢,也严重影响了光电转化效率;热激子衰减的速度,单激子是否在衰减过程中会形成
多激子,也是决定太阳能电池效率的重要因素。\emph{可见,在实际问题中,激子的动力学
  行为往往才是决定器件与材料效率的关键}。

以过渡金属硫化物(TMD)为代表的仅有几个原子厚度的二维材料具有丰富的光学激发态和诸
多其他超越传统体材料的物理性质,已成为构建下一代信息光电器件的重要候选材料。由于
量子限域效应导致介电屏蔽作用减小,激子效应不可忽略。因此,申请人计划利
用\hnamd{}中的$GW{}+{}$real-time BSE功能,选择最熟悉的二维光电材料过渡金属硫化物
(TMD)来系统研究地其中的激子动力学。具体包括\emph{激子的谷间弛豫过程、层间转移过
  程,亮暗激子的转化,以及非辐射复合过程}等,我们还将\emph{考虑杂质、缺陷、应力、
  分子吸附等外界因素}对激子动力学的影响。我们期待能够通过研究揭示电子空穴、自旋轨
道、激子声子等不同的相互作用在激子动力学中的影响,理解动力学现象背后的物理机制,
并争取在此基础上提出调控激子动力学的理论方案,为二维材料器件的设计和性能优化提供
理论指导。

\subsection{技术路线}

申请人计划利用本课题组发展的第一性原理软件包\hnamd{}中的$GW+{}$real-time BSE方法
来研究二维过渡金属硫化物(TMD)中的激子动力学。该方法成功地将单体的含时科
恩--沈(Time-dependent Kohn-Sham)方程推广到含时两体BSE方程(real-time BSE),从而
得到激子波函数的含时演化。由于固体材料在常温的动力学过程中介电环境变化很小,该方
法只进行一次$GW$计算,然后从$W$项中提取介电函数矩阵并在real-time BSE演化中保持不
变,由此在保证精度的前提下大大降低计算量,可以计算上百个原子的激子动力学。通过与
面跳跃方法(Surface Hopping)相结合,该方法利用非绝热耦合来描述激子-声子相互作用;
利用自旋轨道耦合描述自旋的自由度。


\subsection{可行性分析}

申请人此前基于 \hnamd{} 中单粒子图像非绝热分子动力学研究了不同TMD异质结之前的电荷
转移 [\textit{Nano Lett.}, \textbf{17}, 6435 (2017), \textit{Phys. Rev. B},
\textbf{97}, 205417 (2018), \textit{J. Phys. Chem. Lett.}, \textbf{11}, 586
(2020)],得到了一系列重要的结果。这些前期工作说明本人对 TMD 材料的性质非常熟悉,
为本项目的完成提供了必要的条件。另外,申请人在 $GW+{}$real-time BSE 的工作
中[\textit{Sci. Adv.}, \textbf{12}, eabf3759 (2021)],做出了重要贡献(第二作者),
对代码非常熟悉。最后,申请人本人有丰富的设计和编写计算代码的经验,这保证了本项目
的顺利进行。

%%%%%%%%%%%%%%%%%%%%%%%%%%%%%%%%%%%%%%%%%%%%%%%%%%%%%%%%%%%%%%%%%%%%%%%%%%%%%%%%
\section{考虑光场耦合,发展针对材料发光动力学过程的计算方法}

\subsection{科学意义和创新性}

2023年的诺贝尔物理学奖颁发给了Pierre Agostini、Ferenc Krausz和Anne L'Huillier,以
表彰他们``将产生阿秒脉冲的实验方法用于研究物质的电子动力学''。截至目前为止,基于
泵浦--探测的超快激光技术已经与许多凝聚态实验技术结合在一起,为研究凝聚态体系中的
各种维度的超快动力学提供了技术手段。

光场同时也是控制载流子动力学的重要手段。光场与载流子动力学的耦合可以大致分为两个
方面,首先,光场可以激发载流子,人们可以通过控制光子的能量、强度、偏振方向等手段
来控制光激发载流子的能量分布,还可以通过具有手性偏振的光子来控制载流子的自旋;其
次,光场会影响载流子的弛豫过程,电子与空穴可能通过与光场的耦合释放一个光子,发生
辐射复合。在材料发光的过程中,光激发载流子有两条不同的弛豫通道,一条是通过辐射复
合发射光子,而另一条则是通过非辐射复合将能量转移给声子,这两条通道相互竞争,决定
了材料发光的效率,\emph{然而目前并没有合适的第一性原理计算方法来研究}。我们计划
在 \hnamd{} 中加入光场部分,一方面可以用于研究光激发对载流子动力学的调控,另一方
面可以用于研究材料的发光动力学过程,对于发光过程,我们计划给出针对自发辐射与受激
辐射的两种方案。

\subsection{技术路线}

\subsection{可行性分析}
%%%%%%%%%%%%%%%%%%%%%%%%%%%%%%%%%%%%%%%%%%%%%%%%%%%%%%%%%%%%%%%%%%%%%%%%%%%%%%%%